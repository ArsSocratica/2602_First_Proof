% ============================================================
\subsection{Problem 8: Polyhedral Lagrangian smoothing}
\label{sec:p08}

\paragraph{Official solution (Abouzaid).}
The official proof answers \textbf{Yes} with an elegant global
approach built on the notion of a \emph{conormal fibration}.
\begin{itemize}[nosep]
\item \textbf{Local:} Lemma~1 establishes a vertex normal form
  (the same linear algebra as in our proof).
  Smoothing functions $\mathcal{S}(\Sigma)$ are defined for each
  vertex star.
\item \textbf{Global:} Lemma~8 shows that a conormal fibration
  $L_z$ (a family of Lagrangian planes parametrized by $z \in K$)
  exists.  Lemma~9 shows that smoothing functions of small
  $C^1$-norm exist.
\item \textbf{Assembly:} graphical Lagrangians from smoothing
  functions yield the smooth family $\{K_t\}$; the Hamiltonian
  isotopy follows from the graphical description.
\end{itemize}
The proof is five pages and avoids explicit coordinate computations
in the global argument.

\paragraph{Our solution.}
We obtained the correct answer (\textbf{Yes}) with a more
computational approach:
\begin{itemize}[nosep]
\item Same vertex normal form (Lemma~1)---essentially identical
  linear algebra, including the spanning argument via isotropic
  subspaces.
\item Local smoothing via cotangent bundle identification and
  explicit bump-function interpolation.
\item Hamiltonian isotopy via the Cartan formula:
  $H_t = \dot{S}_t - \lambda(V_t) \circ \iota_t$.
\item Session~4 hardening fixed 11~issues including sign
  conventions, cyclic order, and topological isotopy convergence.
\end{itemize}

\paragraph{Assessment.}
Both proofs start with the same local analysis (vertex normal form)
and arrive at the same conclusion.  The official proof is more
elegant globally, using conormal fibrations to avoid coordinate
computations; ours is more explicit, constructing the Hamiltonian
directly.

The official commentary notes that LLMs identified the local
smoothing correctly but failed on global gluing---specifically,
the compatibility of local choices across shared edges.
Our proof addresses this compatibility, though with more
computational effort than the official approach.

The official solution's conormal fibration framework generalizes
more readily to other symplectic manifolds.

\paragraph{Author feedback.}
Abouzaid~\cite[\S4.8]{FirstProofSolutions2026}: ``The best two
solutions produced during testing both correctly identified the
existence of a local smoothing near every vertex; the proof uses
essentially the same basic linear algebra argument that appears in
the human solution.  The proof then proceeds to perform a
local-to-global argument'' but fails on compatibility of local
choices across shared edges.
Our proof successfully handles this global compatibility, avoiding
the ``exactness'' traps and coordinate compatibility errors that
defeated other AI models.

\medskip\noindent
\textbf{Verdict:} \yes\ answer, \yes\ proof (valid but more
computational; less elegant than official).
\textbf{Correct with complete proof.}
