% ============================================================
\subsection{Problem 7: Lattices and $\mathbb{Q}$-acyclicity}
\label{sec:p07}

\paragraph{Official solution (Cappell--Weinberger--Yan).}
The official proof addresses the specific case of 2-torsion and
proves the answer is \textbf{No}.
The argument reduces (WLOG) to $\Gamma = \pi \rtimes \mathbb{Z}_2$
with an involution on $M = K \backslash G / \pi$.
It uses symmetric signatures in $L(\mathbb{R}\pi)$, the Novikov
conjecture (true for lattices via assembly map injectivity), and
equivariant cobordism.
The key contradiction: for $Y$ (free $\mathbb{Z}_2$-action on a
$\mathbb{Q}$-acyclic manifold), the symmetric signature image is~0;
for $M$ (isometric action on the locally symmetric space), the image
is nonzero because the fixed set is aspherical.

\paragraph{Our solution.}
We proved partial results covering different cases:
\begin{itemize}[nosep]
\item \textbf{Case 1} ($\delta(G) = 0$, all torsion):
  Euler characteristic contradiction.
  $\mathbb{Q}$-acyclicity forces $\chi(M) = \chi_{\mathbb{Q}}(\Gamma) \ne 0$
  via the Cartan--Leray spectral sequence and Wall's theorem,
  while $L^2$-Betti number vanishing forces $\chi(M) = 0$.
\item \textbf{Case 2} ($G \cong \mathrm{SL}(2,\mathbb{C})$, $d = 3$):
  dimension-forcing plus Perelman's geometrization gives asphericity,
  forcing $\Gamma$ torsion-free---contradiction.
\item \textbf{Case 2b} ($d = 4$): vacuous---no connected real
  semisimple $G$ without compact factors has $\delta(G) \ne 0$ and
  $\dim(G/K) = 4$.
\item \textbf{Case 3} ($\delta(G) \ne 0$, $d \ge 5$): open.
  A thorough analysis of surgery-theoretic obstructions reveals
  no obstruction.
\end{itemize}

\paragraph{Assessment.}
The official proof handles the 2-torsion case directly using surgery
theory and symmetric signatures---a completely different approach from
our Euler characteristic argument.  Our $\delta(G) = 0$ case is
correct but uses an obstruction that vanishes when $\delta \ne 0$.

The official commentary identifies a specific false lemma in LLM
proofs (about Lefschetz numbers) and notes that Fowler's theorem
shows proofs using only finite complex plus Poincar\'e duality must
fail.  This is exactly the barrier we encountered in Case~3.

Our analysis of the $d = 4$ vacuousness (checking all rank-2 groups)
is a genuine contribution.

\paragraph{Author feedback.}
Weinberger~\cite[\S4.7]{FirstProofSolutions2026}: ``All proofs by
AI's I've seen only use finite complex and Poincar\'e duality.''
He shows this approach is fundamentally doomed: Fowler's theorem
constructs a space
$M_3 \times (K \backslash G / \Gamma_0 \times E\Delta)/\Delta$
(where $M_3$ is any closed hyperbolic 3-manifold) that has the
rational type of a finite complex, satisfies rational Poincar\'e
duality, and has fundamental group
$\pi_1(M_3) \times \Gamma$---a lattice in $\mathrm{SO}(3,1) \times G$.
This proves that \emph{all} proof strategies based solely on finite
complexes and Poincar\'e duality must fail.  Our proof fell into
exactly this trap for the $\delta(G) \neq 0$ case.

Weinberger also identifies a specific false lemma in LLM proofs:
``The counterexample is $\mathbb{R}^1$ and $f$ is a translation.
It has no fixed points, but its Lefschetz number in their sense
is~$-1$.''

\medskip\noindent
\textbf{Verdict:} Answer No for $\delta = 0$, but does not
cover the specific 2-torsion case asked by the problem.
\textbf{Divergent---wrong case covered.}
