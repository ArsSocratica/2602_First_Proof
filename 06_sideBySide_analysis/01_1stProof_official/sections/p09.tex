% ============================================================
\subsection{Problem 9: Quadrilinear determinantal tensors}
\label{sec:p09}

\paragraph{Official solution (Miao--Lerman--Kileel).}
The official proof answers \textbf{Yes} by taking $\mathbf{F}$ to be
the collection of all $5 \times 5$ minors of the four
$3n \times 27n^3$ matrix flattenings of the block tensor~$Q$.
The key observation is a Tucker decomposition
$Q = C \times_1 A \times_2 A \times_3 A \times_4 A$ with
$C_{abcd} = \mathrm{sgn}(abcd)$ (Lemma~1), giving multilinear rank
$\le (4,4,4,4)$, so all $5 \times 5$ minors of any flattening vanish.
The ``only if'' direction uses a three-step argument: normalize
$\lambda$ so that $\lambda_{\alpha 111} = \lambda_{1\beta 11}
= \cdots = 1$, then show that $\lambda$ entries with two ones equal
some constant~$c$, with one one equal~$c^2$, and with zero ones
equal~$c^3$---hence $\lambda$ is rank-1.
Genericity is verified computationally (random numerical instances).

\paragraph{Our solution.}
We constructed a different polynomial map $\mathbf{F}$ consisting of
six families of Pl\"ucker equations $\mathcal{P}_{pq}$, one for each
pair of positions $\{p,q\} \subset \{1,2,3,4\}$.
Each equation has degree~4 (vs.\ degree~5 for the official minors).
\begin{itemize}[nosep]
\item \textbf{Necessity:} the Pl\"ucker identity plus rank-1
  $\lambda$ gives factorization of each equation.
\item \textbf{Sufficiency:} factorization into
  $\Lambda_P \cdot (\Lambda_{S_1} - \Lambda_{S_2}) \cdot
  [Q\text{-bracket}]$; genericity via explicit witness cameras.
\item Numerically verified for $n = 5, 6$.
\end{itemize}

\paragraph{Assessment.}
Both constructions are valid polynomial maps satisfying all three
required properties (independence from~$A$, bounded degree, rank-1
characterization for generic~$A$).
The official approach is more natural from the tensor decomposition
perspective (Tucker rank); ours is more geometric (Pl\"ucker
relations on determinantal varieties).

Our degree bound (4) is tighter than the official (5), which is a
minor advantage.
The official commentary notes that the best LLM answer was
``essentially correct'' using the same $5 \times 5$ minors
approach---our Pl\"ucker approach is a genuinely different
alternative.

\paragraph{Author feedback.}
Kileel~\cite[\S4.9]{FirstProofSolutions2026}: ``The best answer was
essentially correct''---using the same $5 \times 5$ minors approach
as the official solution.
Our Pl\"ucker-elimination approach is a genuinely different
alternative with a tighter degree bound (4 vs.\ 5), providing an
independent verification from a geometric rather than
tensor-algebraic perspective.

\medskip\noindent
\textbf{Verdict:} \yes\ answer, \yes\ proof (alternative
construction with tighter degree bound).
\textbf{Correct with complete proof.}
