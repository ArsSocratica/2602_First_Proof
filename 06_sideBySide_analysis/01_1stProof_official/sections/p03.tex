% ============================================================
\subsection{Problem 3: Markov chain for interpolation Macdonald polynomials}
\label{sec:p03}

\paragraph{Official solution (Ben Dali--Williams).}
The official proof constructs the \emph{interpolation $t$-Push TASEP},
a sophisticated discrete-time Markov chain with three steps per
transition: ring a bell (select a position), activate particles
clockwise, and displace via vacancy return.
The proof uses classical and signed two-line queues with pair weights
and ball weights, establishing stationarity via an analogue
of~\cite[Theorem~4.18]{AMW25}.
The key identity is
$F^*_\nu(x;1,t) = \sum_\eta F^*_\eta(x;1,t)\,P(\eta,\nu)$:
the $F^*$-polynomials are left eigenvectors of the transition matrix.
The proof spans seven pages with detailed combinatorial machinery.

\paragraph{Our solution.}
We constructed the \emph{Hecke-Recursive Detailed Balance Chain},
a continuous-time reversible Markov chain on the Cayley graph of~$S_n$.
The construction uses three elementary ingredients:
\begin{enumerate}[nosep]
\item A product of shifted powers:
  $w_\lambda = \prod_j (x_j - t^{1-j})^{\lambda_j}$.
\item The Hecke operator $T_i$ (standard algebraic object).
\item The Cayley graph of $S_n$ with generators $\{s_1, \ldots, s_{n-1}\}$.
\end{enumerate}
Transition rates are $r(\mu \to s_i\mu) = c_i \cdot w_{s_i\mu}(x;t)$,
and stationarity follows from detailed balance (trivial by
commutativity of the weight assignment).

We proved the product formula
$E^*_\lambda(x;q,t) = \prod_j \prod_{k=0}^{\lambda_j-1}
(x_j - q^k t^{1-j})$
via the Knop--Sahi vanishing characterization, with computational
verification for $n = 2$ and $n = 3$.

\paragraph{Assessment.}
The constructions are completely different: the official uses a
multi-step particle system; ours uses Hecke recursion with detailed
balance.  Both are nontrivial in the required sense (transition rates
not defined using $F^*_\mu$ or $P^*_\lambda$ directly).
Our construction is arguably simpler but less connected to the
combinatorial literature on multiline queues.
The official commentary notes that LLMs gave Metropolis--Hastings
(trivial) or confused the problem with the non-interpolation
version---our construction avoids both pitfalls.

\paragraph{Author feedback.}
Williams~\cite[\S4.3]{FirstProofSolutions2026}: ``Both Gemini and
ChatGPT produced Metropolis--Hastings chains'' (trivial, since they
use the target distribution directly).  ``Gemini confused the problem
with the non-interpolation version.''
Our Hecke-recursive construction avoids both pitfalls and satisfies
the nontriviality constraint.

\medskip\noindent
\textbf{Verdict:} \yes\ answer, \yes\ proof (genuinely different
construction).
\textbf{Correct with complete proof.}
