% ============================================================
\subsection{Problem 2: Rankin--Selberg universal test vectors}
\label{sec:p02}

\paragraph{Official solution (Nelson).}
The proof uses the Godement--Jacquet functional equation as its
main tool.  It constructs an explicit Whittaker function
$W_0(g) = \int_{N_n} \mathbf{1}_{K_n}(xg)\,\psi(x)\,dx$
and relates the $u_Q$-twisted Rankin--Selberg integral to an
integral over $K_1(\mathfrak{q})$ via Schwartz--Bruhat functions
$\beta, \beta^\sharp$ and their Mellin transforms (Lemma~5).
The final result is an explicit formula
$\ell_{\mathrm{RS}}(s,\, u_Q W_0,\, d_Q V) = c\,|Q|^{-n/2}$,
a monomial in $|Q|^s$, hence nonzero for all~$s$.
The proof is six pages and completely self-contained.

\paragraph{Our solution.}
We obtained the same $u_Q$-twist formula
\[
  W\bigl(\diag(g,1)\,u_Q\bigr)
  = \psi^{-1}(Q\,g_{nn})\,W\bigl(\diag(g,1)\bigr)
\]
and used the same monomial-in-$q^{-s}$ strategy.
For universality, we used an existential argument:
JPSS nondegeneracy combined with the observation that the ``bad
locus'' $B_\pi$ depends only on the \emph{inertial} equivalence
class $[\pi]$ (orbit under unramified twists), and inertial classes
are countable.  A $\mathbb{C}$-vector space cannot be a countable
union of proper subspaces, yielding the desired~$W$.

For $n = 1$ ($\GL_2 \times \GL_1$), we gave a fully rigorous
argument via Gauss sums, matching the official solution's
treatment of this base case.

\paragraph{Assessment.}
The official proof is \emph{constructive} (explicit $W_0$, explicit
integral computation); ours is \emph{existential} (dimension counting
over inertial classes).  Both are mathematically valid.
Our inertial class observation---that $B_\pi$ depends only on
$[\pi]$, not on $\pi$ itself---is a genuine insight not present
in the official proof.
The official commentary notes that the best LLM attempt reduced to
the same key integral but failed on the nonvanishing step; our proof
handles this correctly.

\paragraph{Author feedback.}
Nelson~\cite[\S4.2]{FirstProofSolutions2026}: ``The best attempt
[\ldots] reduced to the same key integral but failed on the
nonvanishing step.''
LLMs constructed $W$ depending on~$\pi$ (wrong---universality
requires independence).  Our proof correctly handles universality
via the inertial class countability argument.

\medskip\noindent
\textbf{Verdict:} \yes\ answer, \yes\ proof (different strategy).
\textbf{Correct with complete proof.}
