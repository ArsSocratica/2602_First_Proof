% ============================================================
\subsection{Problem 1: $\Phi^4_3$ measure equivalence under smooth shifts}
\label{sec:p01}

\paragraph{Official solution (Hairer).}
The official proof constructs a separating set
\[
  B_\gamma := \Bigl\{ u \in \mathcal{D}' :
    \lim_{N\to\infty} (\log N)^{-\gamma}
    \bigl\langle H_3(P_N u;\, c_N) + 9\,c_{N,2}\,P_N u,\; \psi
    \bigr\rangle = 0 \Bigr\}
\]
using the Wick cube $H_3$ with polynomial-in-$\log$ scaling
$(\log N)^{-\gamma}$.  The key mechanism is that the mass
renormalization constant $c_{N,2} \gtrsim \log N$ (Lemma~4.3 of the
official solution), so the shifted measure produces a divergent term
$9(\log N)^{-\gamma}\, c_{N,2}\, \langle \psi_N, \psi \rangle
\to \infty$.
The proof uses the Barashkov--Gubinelli decomposition
$u = \Upsilon - \Xi + v$ (Proposition~4.1) and is self-contained
with four detailed lemmas (4.2--4.5).

\paragraph{Our solution.}
We obtained the correct answer (\textbf{No}: $\mu \perp T_\psi^*\mu$
for any nonzero smooth~$\psi$) and constructed a separating set using
Hairer's 2022 note functional~$A_\psi$ with super-exponential
mollification ($\varepsilon_n = e^{-e^n}$) and scaling~$e^{-3n/4}$.
The dominant mechanism is a deterministic mass-shift term
$-b\,e^{n/4}\|\psi\|_{L^2}^2$ arising from the unmollified field.

We additionally analyzed the HKN separating set
$A^{\alpha,\gamma}$~\cite{HKN24} and proved it does \emph{not}
separate $\mu$ from $T_\psi^*\mu$: the smooth shift is absorbed
into the regular remainder.  This analysis is absent from the
official solution.

\paragraph{Assessment.}
Both proofs target mutual singularity via a separating set, but
with different regularization schemes (polynomial-in-$\log$ vs.\
super-exponential).  However, Hairer has confirmed that our proof
is \textbf{incorrect} (see below).

\paragraph{Author feedback.}
Hairer~\cite[\S4.1]{FirstProofSolutions2026}: ``The best solution
produced during testing was by ChatGPT~5.2 Pro [\ldots] it
essentially just quotes the content of [Hai22] without giving a
detailed proof.''

\paragraph{Direct feedback from Hairer on our submission
(February~15, 2026).}
``I had a look at your answer to Q1.  The proof is definitely not
correct: for example the first non-trivial statement, namely `The
renormalized cubic is $O(1)$', is wrong.''

\noindent
In the $\Phi^4_3$ setting, the Wick-ordered cube $:u^3:$ is a
distribution living in a negative Sobolev/Besov regularity space
(roughly $\mathcal{C}^{-1/2-\varepsilon}$), not a bounded function.
Claiming it is $O(1)$ is a fundamental misunderstanding of the
regularity of the objects involved, and this error invalidates the
proof from its first substantive step.  This is a stark illustration
of how AI-generated proofs can produce statements that sound
reasonable but are false in the specific regularity regime of the
problem---and how a non-expert orchestrator cannot catch such errors.

\paragraph{Additional feedback from Gubinelli (February~15, 2026).}
``Besides the problem highlighted by Hairer, it seems the models
induced themselves into believing that the regularisation of the mass
term was the key point (Section~6 in the solution PDF) which is
actually irrelevant (in the original note it was chosen like so for
convenience reasons).  In Section~7 there is a (correct) remark that
the variance of the renormalised cube actually diverges, which is
not-consistent with the model's belief that the renormalised cube
is~$O(1)$.''

\noindent
This reveals a deeper structural problem: the AI models not only made
a false claim about regularity, but also \textbf{misidentified the
mechanism} of the proof.  They fixated on the mass renormalization
term as the key driver of singularity, when in fact it was chosen for
convenience in Hairer's original note and is not the essential
ingredient.  Worse, the models' own Section~7 contains a correct
observation (divergent variance of~$:u^3:$) that directly contradicts
their Section~6 claim ($:u^3:$ is~$O(1)$)---the models failed to
notice this internal inconsistency in their own proof.

\medskip\noindent
\textbf{Verdict:} \yes\ answer,
\textcolor{missred}{\textbf{incorrect proof}}
(Hairer and Gubinelli confirm fundamental errors).
\textbf{Correct answer, wrong proof.}
