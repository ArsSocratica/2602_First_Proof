\documentclass[11pt,a4paper]{article}

\usepackage[utf8]{inputenc}
\usepackage[T1]{fontenc}
\usepackage{amsmath,amssymb,amsthm}
\usepackage{hyperref}
\usepackage{cleveref}
\usepackage{booktabs}
\usepackage{longtable}
\usepackage{enumitem}
\usepackage[margin=1in]{geometry}
\usepackage{xcolor}
\usepackage{microtype}
\usepackage{tabularx}
\usepackage{array}
\usepackage{caption}

\definecolor{matchgreen}{HTML}{2E7D32}
\definecolor{partialorange}{HTML}{E65100}
\definecolor{missred}{HTML}{C62828}

\newtheorem{theorem}{Theorem}[section]
\newtheorem{lemma}[theorem]{Lemma}
\theoremstyle{definition}
\newtheorem{definition}{Definition}[section]
\newtheorem{remark}[theorem]{Remark}

\newcommand{\yes}{\textcolor{matchgreen}{\textbf{Yes}}}
\newcommand{\parti}{\textcolor{partialorange}{\textbf{Partial}}}
\renewcommand{\Phi}{\varPhi}
\newcommand{\eps}{\varepsilon}
\DeclareMathOperator{\diag}{diag}
\DeclareMathOperator{\GL}{GL}

\title{Comparative Analysis of AI-Assisted and Human-Generated\\
  Solutions to the First Proof Challenge}

\author{Mark Dillerop\footnote{Correspondence: \texttt{dillerop@gmail.com}}\\
  \textit{Independent Researcher}}
\date{February 14, 2026}

\begin{document}
\maketitle

\begin{abstract}
The First Proof challenge posed ten research-level mathematical
problems spanning stochastic PDEs, representation theory,
algebraic combinatorics, spectral graph theory, equivariant
topology, manifold theory, symplectic geometry, algebraic
geometry, and numerical linear algebra.  We attempted all ten
using an iterative multi-model AI workflow and submitted solutions
within the eight-day competition window.  This paper provides a
systematic comparison with the official human-generated solutions
released on February~14, 2026.  We define formal assessment
criteria, apply them to each problem, and report the following
outcomes: five problems received correct answers with complete
(though often methodologically distinct) proofs; one received a
valid approach to an open-ended question; one received a correct
answer with an incorrect proof (confirmed by the problem author);
two received correct answers with incomplete proofs; and one
addressed a different case than the one asked.  We analyze the
failure modes, compare our outputs with single-prompt LLM baselines
from the organizers' internal testing, and identify two principal
bottlenecks: cross-field tool discovery and domain-specific
regularity blindness.

\medskip\noindent
\textbf{Disclosure.}  This paper was itself produced using the
same human--AI collaborative workflow it describes.  The author
orchestrated the analysis; the AI systems drafted text,
cross-referenced source files, and identified discrepancies.
All factual claims were verified against the primary sources
(our submitted proofs, the official solutions
document~\cite{FirstProofSolutions2026}, and the problem authors'
commentary).
\end{abstract}

\tableofcontents
\newpage

%% ============================================================
\section{Introduction}\label{sec:intro}

The First Proof challenge~\cite{FirstProof2026} released ten
open research-level mathematical problems on February~5, 2026,
with a submission deadline of February~13.  The problems were
authored by leading researchers in their respective fields and
were designed to test whether current AI systems could produce
genuine mathematical proofs, not merely plausible-sounding
arguments.

We attempted all ten problems using an iterative human--AI
collaborative workflow (described in \S\ref{sec:methodology}),
submitting solutions between February~10 and~12.  On
February~14, the organizers released official solutions and
commentary on LLM-generated outputs from internal
testing~\cite{FirstProofSolutions2026}.  The present paper
provides a systematic, problem-by-problem comparison.

The author is not a mathematician and holds no domain expertise
in any of the ten problem areas.  This is a relevant datum: it
means that the mathematical content of the submitted solutions
was generated entirely by AI systems, with the author serving
as workflow orchestrator rather than mathematical contributor.
The implications of this arrangement are discussed in
\S\ref{sec:discussion}.

%% ============================================================
\section{Related work}\label{sec:related}

AI-assisted theorem proving has progressed rapidly in recent
years.  DeepMind's AlphaProof~\cite{AlphaProof2024} solved
four of six International Mathematical Olympiad 2024 problems
using a reinforcement-learning approach coupled with Lean~4
verification.  Trinh et~al.~\cite{AlphaGeometry2024}
demonstrated that a neuro-symbolic system could solve
Olympiad-level geometry problems.  Romera-Paredes
et~al.~\cite{FunSearch2024} used LLM-guided search to discover
new mathematical constructions in extremal combinatorics.

These systems operate in constrained, well-defined domains
(competition problems, formal geometry, combinatorial search).
The First Proof challenge differs in that its problems are
\emph{research-level}: they require novel proof strategies,
draw on deep domain-specific machinery, and in several cases
connect multiple mathematical fields.  To our knowledge, no
prior work has systematically compared AI-assisted solutions
to research-level problems against expert human solutions
across multiple mathematical disciplines.

The multi-model workflow we employ---using several LLMs in
complementary roles---relates to recent work on LLM
ensembles~\cite{LLMEnsemble2025} and mixture-of-agents
architectures~\cite{MoA2024}, though our approach is
orchestrated manually rather than automated.

%% ============================================================
\section{Methodology}\label{sec:methodology}

\subsection{Workflow}

Five AI systems were used: Claude Opus~4.6 (Anthropic),
ChatGPT~5.2 Pro (OpenAI), Gemini~3.0 Deep Think (Google),
Grok~(xAI), and Perplexity.  Each problem was addressed through
4--12 iterative sessions.  A typical session proceeded as follows:

\begin{enumerate}[nosep]
\item \textbf{Problem ingestion.}  The problem statement and
  relevant context were provided to one or more models.
\item \textbf{Strategy generation.}  Models proposed proof
  strategies; the author selected which to pursue based on
  the models' own assessments of feasibility.
\item \textbf{Proof development.}  The selected model developed
  the proof in detail, with the author feeding intermediate
  outputs to other models for cross-checking.
\item \textbf{Hardening.}  Subsequent sessions identified gaps,
  circularities, or errors in earlier drafts.  Models were
  asked to attack their own proofs and fix identified issues.
\item \textbf{Verification.}  Where feasible, numerical
  experiments (over 900{,}000 tests across all problems) and
  partial Lean~4 formalization (1{,}932 lines total) were used
  to validate claims.
\end{enumerate}

The author's role was limited to orchestration: selecting which
model to engage, routing outputs between models, and deciding
when a proof was sufficiently hardened for submission.  The
author did not set mathematical strategy, evaluate correctness,
or decide proof pivots.

\subsection{Timeline}

Solutions were submitted as follows:
February~10 (P03, P05, P07, P08, P09, P10---six problems);
February~11 (P01, P02, P04, P06---four problems);
February~12 (P06 resubmission).
Per-problem time ranged from approximately 32~minutes (P10) to
over 2~hours (P06).

\subsection{Assessment criteria}\label{sec:criteria}

We evaluate each solution along three dimensions.

\begin{description}[nosep,style=unboxed]
\item[Answer match.]
A solution has \emph{answer match} if its stated conclusion
agrees with the official answer.  For problems admitting
multiple valid answers (e.g., P10, which asks ``explain how''),
any valid method constitutes a match.  Three levels:
\textbf{Yes}~(full agreement),
\textbf{Partial}~(correct direction, weaker claim),
\textbf{Different}~(addresses a different case).

\item[Proof status.]
A proof is \emph{complete} if it establishes the claimed result
without logical gaps, even if the method differs from the
official solution.  It is \emph{incomplete} if the argument has
identified gaps (e.g., covers only special cases).  It is
\emph{wrong case} if it proves a correct statement about a
different mathematical object than the one asked about.

\item[Methodological comparison.]
For each problem, we identify the principal proof technique used
by the official solution and by our solution, and characterize
the relationship: \emph{identical}, \emph{analogous}
(same strategy, different execution), or \emph{distinct}
(fundamentally different approach).
\end{description}

%% ============================================================
\section{Results}\label{sec:results}

\subsection{Summary}

Tables~\ref{tab:answers} and~\ref{tab:approaches} summarize
the comparison across all ten problems.

\begin{table}[ht]
\centering\footnotesize
\caption{Answer comparison.  \textbf{Match}: whether our answer
agrees with the official answer (\S\ref{sec:criteria}).
\textbf{Our proof status}: whether the proof is complete,
incomplete, or addresses the wrong case
(\S\ref{sec:criteria}).}\label{tab:answers}
\begin{tabularx}{\textwidth}{c l p{2.8cm} p{2.8cm} c X}
\toprule
\# & Problem & Official answer & Our answer & Match & Our proof status \\
\midrule
1 & $\Phi^4_3$ shift
  & No (singular)
  & No (singular)
  & \yes & \textcolor{missred}{\textbf{Incorrect}} (Hairer) \\
2 & Rankin--Selberg
  & Yes (constructive)
  & Yes (existential)
  & \yes & Complete \\
3 & Markov--Macdonald
  & Yes ($t$-Push TASEP)
  & Yes (Hecke recursion)
  & \yes & Complete \\
4 & Finite free Stam
  & Yes (all $n$)
  & Yes ($n{\le}3$ only)
  & \parti & Incomplete: gap for $n{\ge}4$ \\
5 & $\mathcal{O}$-slice filtr.
  & Characterization thm
  & Characterization thm
  & \yes & Complete (+ Lean~4) \\
6 & $\eps$-light subsets
  & Yes, $c{=}\eps/42$
  & Yes (no universal $c$)
  & \parti & Incomplete: no universal constant \\
7 & Lattices
  & No (2-torsion)
  & No ($\delta{=}0$ only)
  & \textcolor{missred}{\textbf{Diff.}} & Wrong case \\
8 & Lagrangian smooth.
  & Yes
  & Yes
  & \yes & Complete \\
9 & Quadrilinear tensors
  & Yes (minors, deg~5)
  & Yes (Pl\"ucker, deg~4)
  & \yes & Complete \\
10 & CP-RKHS PCG
  & (Explain how)
  & PCG + subsampled Kron
  & \yes & Complete \\
\bottomrule
\end{tabularx}
\end{table}

\begin{table}[ht]
\centering\footnotesize
\caption{Methodological comparison
(\S\ref{sec:criteria}).}\label{tab:approaches}
\begin{tabularx}{\textwidth}{c l X X}
\toprule
\# & Problem & Official method & Our method \\
\midrule
1 & $\Phi^4_3$
  & $H_3$ Wick cube, $(\log N)^{-\gamma}$ scaling, BG decomposition
  & Hairer's $A_\psi$ functional, super-exponential $e^{-3n/4}$ scaling \\
2 & Rankin--Selberg
  & \emph{Constructive}: explicit $W_0$ via Godement--Jacquet
  & \emph{Existential}: JPSS nondegeneracy + inertial class counting \\
3 & Macdonald
  & $t$-Push TASEP (discrete-time, multiline queues)
  & Hecke-recursive detailed balance (continuous-time, $T_i$-recursion) \\
4 & Stam
  & Jacobian contractivity via Bauschke et al.\ hyperbolic poly.\ convexity
  & Cumulant decomposition, finite free heat equation ($n{\le}3$ only) \\
5 & Slice filtr.
  & Isotropy separation + induction on subgroup lattice
  & Nearly identical; adds Lean~4 formalization (0~\texttt{sorry}) \\
6 & $\eps$-light
  & BSS barrier + leverage score control $\Rightarrow$ $c{=}\eps/42$
  & Multi-bin greedy: $c{=}1/6$ (sparse), $c{=}1/2$ ($K_n$), no universal $c$ \\
7 & Lattices
  & Surgery theory + symmetric signatures + Novikov conjecture
  & Euler characteristic contradiction ($\delta{=}0$ only)---\emph{wrong case} \\
8 & Lagrangian
  & Conormal fibration (elegant, coordinate-free)
  & Hamiltonian via Cartan formula (computational, coordinate-based) \\
9 & Tensors
  & $5{\times}5$ minors of flattenings (degree~5, Tucker rank)
  & Pl\"ucker equations (degree~4, geometric perspective) \\
10 & CP-RKHS
  & PCG + eigendecomposition of $K$ (better conditioning)
  & PCG + subsampled Kronecker matvecs (no eigendecomposition) \\
\bottomrule
\end{tabularx}
\end{table}

\subsection{Aggregate assessment}

Applying the criteria of \S\ref{sec:criteria}, we classify the
ten outcomes into four categories.

\paragraph{Complete proof with correct answer (5/10).}
Problems P02, P03, P05, P08, and P09.  In each case, our
proof reaches the correct conclusion via a valid argument, though
the proof technique often differs substantially from the official
solution (Table~\ref{tab:approaches}).  Of note: P05 nearly
reproduces the official proof structure and additionally includes
Lean~4 formalization with zero \texttt{sorry} axioms; P09
provides a construction of lower degree (4~vs.~5); and P03
offers a genuinely distinct construction based on Hecke operators
rather than the $t$-PushTASEP.

\paragraph{Valid approach to open-ended question (1/10).}
Problem P10 asks ``explain how to use PCG,'' not for a yes/no
answer.  We provided a valid preconditioned conjugate gradient
method using subsampled Kronecker matrix-vector products and a
diagonal preconditioner.  The official solution additionally
transforms the problem via eigendecomposition of the kernel
matrix~$K$, yielding better conditioning.  Kolda~\cite[\S4.10]{FirstProofSolutions2026}
noted that the best internally tested LLM solution was ``correct
and better'' than her provided solution; this refers to the
organizers' internal LLM testing, not to our submission.

\paragraph{Correct answer, incorrect proof (1/10).}
Problem P01 (Hairer): We correctly answered NO (the measures are
mutually singular), but Martin Hairer (the problem author) has
confirmed that our proof is incorrect.  Specifically, the first
non-trivial claim---that the renormalized cubic is $O(1)$---is
wrong.  In the $\Phi^4_3$ setting, the Wick-ordered cube $:u^3:$
is a distribution living in a negative Sobolev/Besov regularity
space, not a bounded quantity.  This invalidates the proof from
its first substantive step.

\paragraph{Correct direction, incomplete proof (2/10).}
\begin{itemize}[nosep]
\item \textbf{P04} (Srivastava): We correctly answered YES and
  proved the result for $n \le 3$, developing 12~theorems and
  20~propositions supported by over 900{,}000 numerical tests.
  However, the official proof establishes the result for
  \emph{all}~$n$ using the Bauschke--G\"uler--Lewis--Sendov
  theorem~\cite{BGLS01} on convexity of eigenvalue functions
  of hyperbolic polynomials.  Our proof has a genuine gap for
  $n \ge 4$.
\item \textbf{P06} (Spielman): We correctly answered YES and
  proved several partial cases ($c = 1/6$ for sparse graphs,
  $c = 1/3$ for bounded degeneracy, $c = 1/2$ for~$K_n$).
  The official proof provides a universal constant
  $c = \eps/42$ via BSS barrier functions~\cite{BSS12}.  We
  could not close the general case.
\end{itemize}

\paragraph{Divergent---wrong case addressed (1/10).}
Problem P07 (Weinberger) asks specifically about
\textbf{2-torsion} in the fundamental group.  The official
solution proves the answer is NO using surgery theory,
symmetric signatures in $L(\mathbb{R}\pi)$, and the Novikov
conjecture for lattices.  Our proof establishes NO for a
\emph{different} case ($\delta(G) = 0$, covering groups of
vanishing fundamental rank) and leaves the $\delta(G) \ne 0$
case open---which is precisely where the 2-torsion case
resides.  Weinberger's commentary~\cite[\S4.7]{FirstProofSolutions2026}
further notes that Fowler's theorem shows \emph{all} proof
strategies based on finite complexes and Poincar\'e duality
must fail for this problem---exactly the approach we employed.

%% ============================================================
\section{Problem-by-problem comparison}\label{sec:problems}

% ============================================================
\subsection{Problem 1: $\Phi^4_3$ measure equivalence under smooth shifts}
\label{sec:p01}

\paragraph{Official solution (Hairer).}
The official proof constructs a separating set
\[
  B_\gamma := \Bigl\{ u \in \mathcal{D}' :
    \lim_{N\to\infty} (\log N)^{-\gamma}
    \bigl\langle H_3(P_N u;\, c_N) + 9\,c_{N,2}\,P_N u,\; \psi
    \bigr\rangle = 0 \Bigr\}
\]
using the Wick cube $H_3$ with polynomial-in-$\log$ scaling
$(\log N)^{-\gamma}$.  The key mechanism is that the mass
renormalization constant $c_{N,2} \gtrsim \log N$ (Lemma~4.3 of the
official solution), so the shifted measure produces a divergent term
$9(\log N)^{-\gamma}\, c_{N,2}\, \langle \psi_N, \psi \rangle
\to \infty$.
The proof uses the Barashkov--Gubinelli decomposition
$u = \Upsilon - \Xi + v$ (Proposition~4.1) and is self-contained
with four detailed lemmas (4.2--4.5).

\paragraph{Our solution.}
We obtained the correct answer (\textbf{No}: $\mu \perp T_\psi^*\mu$
for any nonzero smooth~$\psi$) and constructed a separating set using
Hairer's 2022 note functional~$A_\psi$ with super-exponential
mollification ($\varepsilon_n = e^{-e^n}$) and scaling~$e^{-3n/4}$.
The dominant mechanism is a deterministic mass-shift term
$-b\,e^{n/4}\|\psi\|_{L^2}^2$ arising from the unmollified field.

We additionally analyzed the HKN separating set
$A^{\alpha,\gamma}$~\cite{HKN24} and proved it does \emph{not}
separate $\mu$ from $T_\psi^*\mu$: the smooth shift is absorbed
into the regular remainder.  This analysis is absent from the
official solution.

\paragraph{Assessment.}
Both proofs establish mutual singularity via a separating set, but
with different regularization schemes (polynomial-in-$\log$ vs.\
super-exponential).  The official proof is more self-contained; ours
relies more on citing Hairer's note.  Our HKN failure analysis is a
genuine contribution.

\paragraph{Author feedback.}
Hairer~\cite[\S4.1]{FirstProofSolutions2026}: ``The best solution
produced during testing was by ChatGPT~5.2 Pro [\ldots] it
essentially just quotes the content of [Hai22] without giving a
detailed proof.''
Our proof goes significantly beyond this by providing detailed
verification and analyzing why the HKN framework fails.

\medskip\noindent
\textbf{Verdict:} \yes\ answer, \yes\ proof (different mechanism).
\textbf{Correct with complete proof.}

% ============================================================
\subsection{Problem 2: Rankin--Selberg universal test vectors}
\label{sec:p02}

\paragraph{Official solution (Nelson).}
The proof uses the Godement--Jacquet functional equation as its
main tool.  It constructs an explicit Whittaker function
$W_0(g) = \int_{N_n} \mathbf{1}_{K_n}(xg)\,\psi(x)\,dx$
and relates the $u_Q$-twisted Rankin--Selberg integral to an
integral over $K_1(\mathfrak{q})$ via Schwartz--Bruhat functions
$\beta, \beta^\sharp$ and their Mellin transforms (Lemma~5).
The final result is an explicit formula
$\ell_{\mathrm{RS}}(s,\, u_Q W_0,\, d_Q V) = c\,|Q|^{-n/2}$,
a monomial in $|Q|^s$, hence nonzero for all~$s$.
The proof is six pages and completely self-contained.

\paragraph{Our solution.}
We obtained the same $u_Q$-twist formula
\[
  W\bigl(\diag(g,1)\,u_Q\bigr)
  = \psi^{-1}(Q\,g_{nn})\,W\bigl(\diag(g,1)\bigr)
\]
and used the same monomial-in-$q^{-s}$ strategy.
For universality, we used an existential argument:
JPSS nondegeneracy combined with the observation that the ``bad
locus'' $B_\pi$ depends only on the \emph{inertial} equivalence
class $[\pi]$ (orbit under unramified twists), and inertial classes
are countable.  A $\mathbb{C}$-vector space cannot be a countable
union of proper subspaces, yielding the desired~$W$.

For $n = 1$ ($\GL_2 \times \GL_1$), we gave a fully rigorous
argument via Gauss sums, matching the official solution's
treatment of this base case.

\paragraph{Assessment.}
The official proof is \emph{constructive} (explicit $W_0$, explicit
integral computation); ours is \emph{existential} (dimension counting
over inertial classes).  Both are mathematically valid.
Our inertial class observation---that $B_\pi$ depends only on
$[\pi]$, not on $\pi$ itself---is a genuine insight not present
in the official proof.
The official commentary notes that the best LLM attempt reduced to
the same key integral but failed on the nonvanishing step; our proof
handles this correctly.

\paragraph{Author feedback.}
Nelson~\cite[\S4.2]{FirstProofSolutions2026}: ``The best attempt
[\ldots] reduced to the same key integral but failed on the
nonvanishing step.''
LLMs constructed $W$ depending on~$\pi$ (wrong---universality
requires independence).  Our proof correctly handles universality
via the inertial class countability argument.

\medskip\noindent
\textbf{Verdict:} \yes\ answer, \yes\ proof (different strategy).
\textbf{Correct with complete proof.}

% ============================================================
\subsection{Problem 3: Markov chain for interpolation Macdonald polynomials}
\label{sec:p03}

\paragraph{Official solution (Ben Dali--Williams).}
The official proof constructs the \emph{interpolation $t$-Push TASEP},
a sophisticated discrete-time Markov chain with three steps per
transition: ring a bell (select a position), activate particles
clockwise, and displace via vacancy return.
The proof uses classical and signed two-line queues with pair weights
and ball weights, establishing stationarity via an analogue
of~\cite[Theorem~4.18]{AMW25}.
The key identity is
$F^*_\nu(x;1,t) = \sum_\eta F^*_\eta(x;1,t)\,P(\eta,\nu)$:
the $F^*$-polynomials are left eigenvectors of the transition matrix.
The proof spans seven pages with detailed combinatorial machinery.

\paragraph{Our solution.}
We constructed the \emph{Hecke-Recursive Detailed Balance Chain},
a continuous-time reversible Markov chain on the Cayley graph of~$S_n$.
The construction uses three elementary ingredients:
\begin{enumerate}[nosep]
\item A product of shifted powers:
  $w_\lambda = \prod_j (x_j - t^{1-j})^{\lambda_j}$.
\item The Hecke operator $T_i$ (standard algebraic object).
\item The Cayley graph of $S_n$ with generators $\{s_1, \ldots, s_{n-1}\}$.
\end{enumerate}
Transition rates are $r(\mu \to s_i\mu) = c_i \cdot w_{s_i\mu}(x;t)$,
and stationarity follows from detailed balance (trivial by
commutativity of the weight assignment).

We proved the product formula
$E^*_\lambda(x;q,t) = \prod_j \prod_{k=0}^{\lambda_j-1}
(x_j - q^k t^{1-j})$
via the Knop--Sahi vanishing characterization, with computational
verification for $n = 2$ and $n = 3$.

\paragraph{Assessment.}
The constructions are completely different: the official uses a
multi-step particle system; ours uses Hecke recursion with detailed
balance.  Both are nontrivial in the required sense (transition rates
not defined using $F^*_\mu$ or $P^*_\lambda$ directly).
Our construction is arguably simpler but less connected to the
combinatorial literature on multiline queues.
The official commentary notes that LLMs gave Metropolis--Hastings
(trivial) or confused the problem with the non-interpolation
version---our construction avoids both pitfalls.

\paragraph{Author feedback.}
Williams~\cite[\S4.3]{FirstProofSolutions2026}: ``Both Gemini and
ChatGPT produced Metropolis--Hastings chains'' (trivial, since they
use the target distribution directly).  ``Gemini confused the problem
with the non-interpolation version.''
Our Hecke-recursive construction avoids both pitfalls and satisfies
the nontriviality constraint.

\medskip\noindent
\textbf{Verdict:} \yes\ answer, \yes\ proof (genuinely different
construction).
\textbf{Correct with complete proof.}

% ============================================================
\subsection{Problem 4: Finite free Stam inequality}
\label{sec:p04}

\paragraph{Official solution (Garza Vargas--Srivastava--Stier).}
The official proof is \emph{complete for all~$n$} and follows a
three-step strategy \`a la Blachman:
\begin{enumerate}[nosep]
\item \textbf{Score vectors as derivatives} (Lemma~1.1):
  under reverse heat flow $p_t = p \boxplus_n \sqrt{t}\,\mathrm{He}_n$,
  the roots satisfy $\alpha_i'(0) = J_n(\alpha)_i$.
\item \textbf{Jacobian contractivity} (Proposition~2.1):
  $\|J_{\boxplus_n}(u,v)\|^2 \le \|u\|^2 + \|v\|^2$ for
  $u, v \perp \mathbf{1}_n$.
\item \textbf{Blachman's argument}: choose
  $a = 1/\|J_n(\alpha)\|$, $b = 1/\|J_n(\beta)\|$, and combine.
\end{enumerate}
The key insight is the use of the Bauschke--G\"uler--Lewis--Sendov
theorem~\cite{BGLS01} on convexity of eigenvalue functions of
hyperbolic polynomials: the Hessians $H^{(i)}_{\boxplus_n}$ satisfy
$\sum c_i H^{(i)} \succeq 0$ for ordered~$c_i$, via hyperbolicity
of the finite free convolution polynomial.

\paragraph{Our solution.}
We obtained the correct answer (\textbf{Yes}) and proved the
inequality for $n = 2$ (exact equality) and $n = 3$ (explicit formula).
We developed extensive supporting theory:
\begin{itemize}[nosep]
\item Finite free heat equation (Theorem~6):
  $dp_t/dt = -p_t''/2$.
\item Root velocity equals score (Corollary~6.1):
  $\lambda_i'(t) = p''(\lambda_i)/(2p'(\lambda_i))$.
\item Finite de Bruijn identity (Theorem~7):
  $d/dt \log \Delta(p_t) = \Phi_n(p_t)$.
\item $\Phi_n$ monotonicity (Theorem~8):
  $\Phi' = -2\sum_{i<j} (h_i - h_j)^2/(\lambda_i - \lambda_j)^2 \le 0$.
\item $J$-concavity for $n = 3$ (Theorem~11):
  $J(t) = 1/\Phi(p_t)$ is concave.
\end{itemize}
In total: 12~theorems, 20+~propositions, 900{,}000+ numerical tests
with zero violations.  The general case remained open: the gap was
the generalized concavity of a remainder term $r(\sigma)$ along
contraction paths.

\paragraph{Assessment.}
This is the largest gap between our work and the official solution.
The official proof is complete; ours covers only $n \le 3$.
Our approach (cumulant decomposition, explicit remainder analysis)
is fundamentally different from theirs (Jacobian contractivity via
hyperbolic polynomial theory).

The missing ingredient---the Bauschke et~al.\ convexity
theorem---connects real algebraic geometry to information theory
in a non-obvious way.  Despite extensive search (8+ sessions),
we did not discover this connection.  However, our finite free
heat equation, de Bruijn identity, and score--root identities are
genuine new results not present in the official solution.

The official commentary describes exactly the difficulty we
encountered: the LLM tried Blachman's approach but could not find
the joint probability space analogue.

\paragraph{Author feedback.}
Srivastava~\cite[\S4.4]{FirstProofSolutions2026}: ``The LLM tried
Blachman's approach but couldn't find the joint probability space
analogue.''
This is exactly the difficulty we encountered.  The official authors
resolved it via the Bauschke--G\"uler--Lewis--Sendov
theorem~\cite{BGLS01} on hyperbolic polynomial convexity---a
connection between real algebraic geometry and information theory
that we did not discover despite 8+ sessions of exploration.

\medskip\noindent
\textbf{Verdict:} \yes\ answer, \parti\ proof ($n \le 3$ only;
official proof is complete for all~$n$).
\textbf{Correct direction, incomplete proof.}

% ============================================================
\subsection{Problem 5: Equivariant $\mathcal{O}$-slice filtration}
\label{sec:p05}

\paragraph{Official solution (Blumberg--Hill--Lawson).}
The official proof defines the $\mathcal{O}$-slice filtration via
localizing subcategories generated by
$G_+ \wedge_H N_T S^1$ for admissible~$T$, and introduces the
characteristic function
$\chi_{\mathcal{O}}(H) = \min\{K : K \to H\}$.
The main result (Theorem~2.7) states:
$E \in \tau^{\mathcal{O}}_{\ge n}$ if and only if
$[H : \chi_{\mathcal{O}}(H)] \cdot \mathrm{gconn}(E)(H) \ge n$
for all~$H$.
The forward direction uses Lemma~2.3 (restriction to generators);
the reverse uses isotropy separation, Lemma~2.5, Lemma~2.6
(geometric Mackey functors), and induction on the subgroup lattice.

\paragraph{Our solution.}
Our Theorem~3.1 states: $X$ is $\mathcal{O}$-slice $\ge n$ if and
only if $\Phi^H X$ is $(\lfloor n/|H| \rfloor - 1)$-connected for
all $H \in \mathcal{F}_{\mathcal{O}}$.
The forward direction uses strong induction on~$|H|$ and the
Wirthm\"uller isomorphism; the reverse uses isotropy separation
and induction on the subgroup lattice.

We additionally produced a Lean~4 formalization (0~\texttt{sorry})
verifying the combinatorial and arithmetic skeleton: transfer system
axioms, dimension bookkeeping, and the strong induction structure.

\paragraph{Assessment.}
This is the \emph{closest match} to the official solution among all
ten problems.  Both proofs use isotropy separation and induction on
the subgroup lattice.  The connectivity bound formulas differ
slightly in notation ($[H:\chi_{\mathcal{O}}(H)]$ vs.\ $|H|$ with
restriction to $\mathcal{F}_{\mathcal{O}}$) but are equivalent when
$\chi_{\mathcal{O}}(H) = e$.

Both use the same key lemmas: Wirthm\"uller isomorphism, compactness
of slice cells, geometric fixed point computation.
The official commentary notes that LLMs got the statement essentially
correct but proofs were ``sketched or slightly garbled''---our proof
is fully detailed.

Our Lean~4 formalization adds a layer of verification not present
in the official release.

\paragraph{Author feedback.}
Blumberg~\cite[\S4.5]{FirstProofSolutions2026}: ``The LLMs got the
statement essentially correct but proofs were `sketched or slightly
garbled'.''
Our proof is fully detailed with all steps, and our Lean~4
formalization (0~\texttt{sorry}) adds a layer of verification not
present in any other submission.

\medskip\noindent
\textbf{Verdict:} \yes\ answer, \yes\ proof (nearly identical
structure, plus Lean~4).
\textbf{Correct with complete proof.}

% ============================================================
\subsection{Problem 6: $\eps$-light subsets}
\label{sec:p06}

\paragraph{Official solution (Spielman).}
The official proof is \emph{complete} with constant $c = 1/42$
(i.e., $|S| \ge \eps n / 42$).
It uses a greedy construction, adding vertices one at a time,
controlled by a modified BSS barrier function~\cite{BSS12}:
\[
  \Phi^u_\sigma(A)
  = \sum_{i=1}^{\sigma} \frac{1}{u - \lambda_i(A)}
\]
tracking only the top-$\sigma$ eigenvalues.
The key lemma (Lemma~1.2) shows that at each step, one can find a
vertex~$t$ such that both the barrier $\Phi$ remains bounded and
the leverage score $\ell(S) \le 4|S|$ is maintained.
The proof uses the Ky Fan eigenvalue inequality and the
Sherman--Morrison--Woodbury formula.

\paragraph{Our solution.}
We obtained the correct answer (\textbf{Yes}) but with an
\emph{incomplete} proof.  Our proved results include:
\begin{itemize}[nosep]
\item $c \le 1/2$ (tight upper bound, via disjoint cliques).
\item $c = 1/6$ for sparse graphs ($\bar{d} \le 6/\eps - 1$).
\item $c = 1/2$ for complete graphs $K_n$ (direct spectral argument).
\item $c = 1/3$ for graphs with degeneracy $< \lceil 3/\eps \rceil$.
\item Extensive computational verification: multi-bin greedy with
  $k = \lceil 2/\eps \rceil$ bins never gets stuck across 16+ graph
  families, $n \le 100$, $\eps \in \{0.1, \ldots, 0.5\}$.
\end{itemize}
The gap: we could not prove that greedy always finds a bin with
$\mu_{\max} < 1$ at every step.

\paragraph{Assessment.}
The official proof uses a BSS-type barrier from spectral
sparsification---a specific technique from theoretical computer
science that we did not employ.  Our approach (multi-bin greedy with
effective resistance analysis) is different and more computational.

The official commentary notes that Gemini's proof was vague and
GPT-Pro only gave the upper bound $c \le 1/2$.  Our partial results
(four proved theorems, extensive computation) go well beyond both
LLM attempts.

\paragraph{Author feedback.}
Spielman~\cite[\S4.6]{FirstProofSolutions2026}: ``Gemini asserted
that it presented a proof [\ldots] But, after some correct
statements, it presented a very vague explanation of how the proof
could be finished.  To me, it seems unlikely that the approach can
be turned into a correct proof.''
ChatGPT~5.2 Pro only offered the upper bound $c \le 1/2$.
Our partial results (four proved theorems, log-det barrier, Barbell
graph identification, extensive computation across 16+ graph
families) go substantially beyond both LLM attempts.

\medskip\noindent
\textbf{Verdict:} \yes\ answer, \parti\ proof (no universal
constant; official proves $c = \eps/42$).
\textbf{Correct direction, incomplete proof.}

% ============================================================
\subsection{Problem 7: Lattices and $\mathbb{Q}$-acyclicity}
\label{sec:p07}

\paragraph{Official solution (Cappell--Weinberger--Yan).}
The official proof addresses the specific case of 2-torsion and
proves the answer is \textbf{No}.
The argument reduces (WLOG) to $\Gamma = \pi \rtimes \mathbb{Z}_2$
with an involution on $M = K \backslash G / \pi$.
It uses symmetric signatures in $L(\mathbb{R}\pi)$, the Novikov
conjecture (true for lattices via assembly map injectivity), and
equivariant cobordism.
The key contradiction: for $Y$ (free $\mathbb{Z}_2$-action on a
$\mathbb{Q}$-acyclic manifold), the symmetric signature image is~0;
for $M$ (isometric action on the locally symmetric space), the image
is nonzero because the fixed set is aspherical.

\paragraph{Our solution.}
We proved partial results covering different cases:
\begin{itemize}[nosep]
\item \textbf{Case 1} ($\delta(G) = 0$, all torsion):
  Euler characteristic contradiction.
  $\mathbb{Q}$-acyclicity forces $\chi(M) = \chi_{\mathbb{Q}}(\Gamma) \ne 0$
  via the Cartan--Leray spectral sequence and Wall's theorem,
  while $L^2$-Betti number vanishing forces $\chi(M) = 0$.
\item \textbf{Case 2} ($G \cong \mathrm{SL}(2,\mathbb{C})$, $d = 3$):
  dimension-forcing plus Perelman's geometrization gives asphericity,
  forcing $\Gamma$ torsion-free---contradiction.
\item \textbf{Case 2b} ($d = 4$): vacuous---no connected real
  semisimple $G$ without compact factors has $\delta(G) \ne 0$ and
  $\dim(G/K) = 4$.
\item \textbf{Case 3} ($\delta(G) \ne 0$, $d \ge 5$): open.
  A thorough analysis of surgery-theoretic obstructions reveals
  no obstruction.
\end{itemize}

\paragraph{Assessment.}
The official proof handles the 2-torsion case directly using surgery
theory and symmetric signatures---a completely different approach from
our Euler characteristic argument.  Our $\delta(G) = 0$ case is
correct but uses an obstruction that vanishes when $\delta \ne 0$.

The official commentary identifies a specific false lemma in LLM
proofs (about Lefschetz numbers) and notes that Fowler's theorem
shows proofs using only finite complex plus Poincar\'e duality must
fail.  This is exactly the barrier we encountered in Case~3.

Our analysis of the $d = 4$ vacuousness (checking all rank-2 groups)
is a genuine contribution.

\paragraph{Author feedback.}
Weinberger~\cite[\S4.7]{FirstProofSolutions2026}: ``All proofs by
AI's I've seen only use finite complex and Poincar\'e duality.''
He shows this approach is fundamentally doomed: Fowler's theorem
constructs a space
$M_3 \times (K \backslash G / \Gamma_0 \times E\Delta)/\Delta$
(where $M_3$ is any closed hyperbolic 3-manifold) that has the
rational type of a finite complex, satisfies rational Poincar\'e
duality, and has fundamental group
$\pi_1(M_3) \times \Gamma$---a lattice in $\mathrm{SO}(3,1) \times G$.
This proves that \emph{all} proof strategies based solely on finite
complexes and Poincar\'e duality must fail.  Our proof fell into
exactly this trap for the $\delta(G) \neq 0$ case.

Weinberger also identifies a specific false lemma in LLM proofs:
``The counterexample is $\mathbb{R}^1$ and $f$ is a translation.
It has no fixed points, but its Lefschetz number in their sense
is~$-1$.''

\medskip\noindent
\textbf{Verdict:} Answer No for $\delta = 0$, but does not
cover the specific 2-torsion case asked by the problem.
\textbf{Divergent---wrong case covered.}

% ============================================================
\subsection{Problem 8: Polyhedral Lagrangian smoothing}
\label{sec:p08}

\paragraph{Official solution (Abouzaid).}
The official proof answers \textbf{Yes} with an elegant global
approach built on the notion of a \emph{conormal fibration}.
\begin{itemize}[nosep]
\item \textbf{Local:} Lemma~1 establishes a vertex normal form
  (the same linear algebra as in our proof).
  Smoothing functions $\mathcal{S}(\Sigma)$ are defined for each
  vertex star.
\item \textbf{Global:} Lemma~8 shows that a conormal fibration
  $L_z$ (a family of Lagrangian planes parametrized by $z \in K$)
  exists.  Lemma~9 shows that smoothing functions of small
  $C^1$-norm exist.
\item \textbf{Assembly:} graphical Lagrangians from smoothing
  functions yield the smooth family $\{K_t\}$; the Hamiltonian
  isotopy follows from the graphical description.
\end{itemize}
The proof is five pages and avoids explicit coordinate computations
in the global argument.

\paragraph{Our solution.}
We obtained the correct answer (\textbf{Yes}) with a more
computational approach:
\begin{itemize}[nosep]
\item Same vertex normal form (Lemma~1)---essentially identical
  linear algebra, including the spanning argument via isotropic
  subspaces.
\item Local smoothing via cotangent bundle identification and
  explicit bump-function interpolation.
\item Hamiltonian isotopy via the Cartan formula:
  $H_t = \dot{S}_t - \lambda(V_t) \circ \iota_t$.
\item Session~4 hardening fixed 11~issues including sign
  conventions, cyclic order, and topological isotopy convergence.
\end{itemize}

\paragraph{Assessment.}
Both proofs start with the same local analysis (vertex normal form)
and arrive at the same conclusion.  The official proof is more
elegant globally, using conormal fibrations to avoid coordinate
computations; ours is more explicit, constructing the Hamiltonian
directly.

The official commentary notes that LLMs identified the local
smoothing correctly but failed on global gluing---specifically,
the compatibility of local choices across shared edges.
Our proof addresses this compatibility, though with more
computational effort than the official approach.

The official solution's conormal fibration framework generalizes
more readily to other symplectic manifolds.

\paragraph{Author feedback.}
Abouzaid~\cite[\S4.8]{FirstProofSolutions2026}: ``The best two
solutions produced during testing both correctly identified the
existence of a local smoothing near every vertex; the proof uses
essentially the same basic linear algebra argument that appears in
the human solution.  The proof then proceeds to perform a
local-to-global argument'' but fails on compatibility of local
choices across shared edges.
Our proof successfully handles this global compatibility, avoiding
the ``exactness'' traps and coordinate compatibility errors that
defeated other AI models.

\medskip\noindent
\textbf{Verdict:} \yes\ answer, \yes\ proof (valid but more
computational; less elegant than official).
\textbf{Correct with complete proof.}

% ============================================================
\subsection{Problem 9: Quadrilinear determinantal tensors}
\label{sec:p09}

\paragraph{Official solution (Miao--Lerman--Kileel).}
The official proof answers \textbf{Yes} by taking $\mathbf{F}$ to be
the collection of all $5 \times 5$ minors of the four
$3n \times 27n^3$ matrix flattenings of the block tensor~$Q$.
The key observation is a Tucker decomposition
$Q = C \times_1 A \times_2 A \times_3 A \times_4 A$ with
$C_{abcd} = \mathrm{sgn}(abcd)$ (Lemma~1), giving multilinear rank
$\le (4,4,4,4)$, so all $5 \times 5$ minors of any flattening vanish.
The ``only if'' direction uses a three-step argument: normalize
$\lambda$ so that $\lambda_{\alpha 111} = \lambda_{1\beta 11}
= \cdots = 1$, then show that $\lambda$ entries with two ones equal
some constant~$c$, with one one equal~$c^2$, and with zero ones
equal~$c^3$---hence $\lambda$ is rank-1.
Genericity is verified computationally (random numerical instances).

\paragraph{Our solution.}
We constructed a different polynomial map $\mathbf{F}$ consisting of
six families of Pl\"ucker equations $\mathcal{P}_{pq}$, one for each
pair of positions $\{p,q\} \subset \{1,2,3,4\}$.
Each equation has degree~4 (vs.\ degree~5 for the official minors).
\begin{itemize}[nosep]
\item \textbf{Necessity:} the Pl\"ucker identity plus rank-1
  $\lambda$ gives factorization of each equation.
\item \textbf{Sufficiency:} factorization into
  $\Lambda_P \cdot (\Lambda_{S_1} - \Lambda_{S_2}) \cdot
  [Q\text{-bracket}]$; genericity via explicit witness cameras.
\item Numerically verified for $n = 5, 6$.
\end{itemize}

\paragraph{Assessment.}
Both constructions are valid polynomial maps satisfying all three
required properties (independence from~$A$, bounded degree, rank-1
characterization for generic~$A$).
The official approach is more natural from the tensor decomposition
perspective (Tucker rank); ours is more geometric (Pl\"ucker
relations on determinantal varieties).

Our degree bound (4) is tighter than the official (5), which is a
minor advantage.
The official commentary notes that the best LLM answer was
``essentially correct'' using the same $5 \times 5$ minors
approach---our Pl\"ucker approach is a genuinely different
alternative.

\paragraph{Author feedback.}
Kileel~\cite[\S4.9]{FirstProofSolutions2026}: ``The best answer was
essentially correct''---using the same $5 \times 5$ minors approach
as the official solution.
Our Pl\"ucker-elimination approach is a genuinely different
alternative with a tighter degree bound (4 vs.\ 5), providing an
independent verification from a geometric rather than
tensor-algebraic perspective.

\medskip\noindent
\textbf{Verdict:} \yes\ answer, \yes\ proof (alternative
construction with tighter degree bound).
\textbf{Correct with complete proof.}

% ============================================================
\subsection{Problem 10: CP-RKHS iterative solver}
\label{sec:p10}

\paragraph{Official solution (Brust--Kolda).}
The official proof designs a preconditioned conjugate gradient (PCG)
solver for the mode-$k$ RKHS subproblem of CP-HiFi tensor
decomposition.  The key innovation is a \emph{change of variables}
via the eigendecomposition $K = UDU^T$: defining
$\bar{F} = S^T(Z \otimes UD)$ and solving for $\bar{W} = U^T W$
yields a better-conditioned system.
Three lemmas establish efficient matrix--vector products via
row-wise Kronecker structure:
\begin{itemize}[nosep]
\item $Cx = (A \ast BX)\mathbf{1}_r$ at cost $O(q(r+n))$.
\item $C^T v = \mathrm{vec}(B^T \diag(v)\, A)$.
\item $\diag(C^T C) = \mathrm{vec}((B \ast B)^T (A \ast A))$.
\end{itemize}
A diagonal preconditioner is constructed from
$\diag(\bar{F}^T \bar{F}) + \lambda(I_r \otimes D) + \rho I_{rn}$.
Total cost: $O(qn^2 + qr^2 + qnrp)$ per CG iteration, vs.\
$O(qn^2r^2 + n^3r^3)$ for direct Cholesky.

\paragraph{Our solution.}
We proved $\mathbf{H}$ is symmetric positive definite
(Proposition~2.1), designed an efficient matrix--vector product via
subsampled Kronecker structure (reshaping $\mathbf{v} = \mathrm{vec}(V)$
and computing $(Z \otimes K)^T S S^T (Z \otimes K)\,\mathbf{v}$
entry-by-entry for observed indices), and proposed a complete-data
Kronecker preconditioner $\mathbf{P} = \Gamma \otimes K^2 + \lambda(I_r \otimes K)$
with simultaneous Kronecker-diagonalizable structure.
Complexity analysis was provided.

\paragraph{Assessment.}
Both solutions use PCG with efficient matrix--vector products
avoiding $O(N)$ computation.  The official solution uses a diagonal
preconditioner after an eigendecomposition change of variables;
our solution uses a complete-data Kronecker preconditioner with
simultaneous diagonalization in the Kronecker eigenbasis.
The official solution's eigendecomposition transformation is the
``advanced'' step that Kolda noted she would be impressed if AI
could produce.

Both use the row-wise Kronecker structure for efficient matvecs.
The official commentary notes that the best LLM solution was
``correct and better than the solution I provided,'' citing the
subsampled Kronecker matvec idea from prior literature
(arXiv:1601.01507).

\paragraph{Author feedback.}
Kolda~\cite[\S4.10]{FirstProofSolutions2026}: ``The best LLM
solution was correct and better than the solution I provided''
---referring to the internally tested LLM (Feb~4--5), not our
submission.  Kolda noted she would be ``impressed if the AI can''
discover the eigendecomposition transformation; our solution does
\emph{not} include this step.  Our subsampled Kronecker matvec
approach draws on prior literature (arXiv:1601.01507) and is a
valid PCG method, but does not achieve the full transformation
the official solution provides.

\medskip\noindent
\textbf{Verdict:} \yes\ answer, \yes\ proof (valid PCG approach;
does not include the eigendecomposition transformation).
\textbf{Correct---valid approach to an open-ended question.}


%% ============================================================
\section{Discussion}\label{sec:discussion}

\subsection{Strengths of the AI-assisted approach}

In five of ten problems, the iterative multi-model workflow
produced complete, correct proofs.  Several of these exhibit
methodological independence from the official solutions:
P03 constructs a Hecke-recursive chain rather than a
$t$-PushTASEP; P09 uses Pl\"ucker equations of degree~4
rather than $5{\times}5$ minors of degree~5; and P02 gives
an existential argument via inertial class counting rather
than an explicit construction.  This methodological diversity
suggests that the AI systems are not merely reproducing known
approaches but can synthesize novel proof strategies within
a given mathematical field.

\subsection{Failure mode analysis}

The four problems with incorrect, incomplete, or divergent
results (P01, P04, P06, P07) reveal two distinct failure modes.
The first is \textbf{cross-field tool discovery}.  Three problems
share a common structural feature: the official proof requires
importing a specific tool from an \emph{adjacent} mathematical
field that is not suggested by the problem statement itself.

\begin{itemize}[nosep]
\item \textbf{P04} required the Bauschke--G\"uler--Lewis--Sendov
  theorem~\cite{BGLS01}, connecting real algebraic geometry
  (hyperbolic polynomials) to information theory (entropy power
  inequalities).
\item \textbf{P06} required BSS barrier functions~\cite{BSS12}
  from spectral sparsification theory, applied to a graph
  partitioning problem.
\item \textbf{P07} required surgery-theoretic tools (symmetric
  signatures, the Novikov conjecture for lattices, equivariant
  cobordism) applied to a question about lattice topology.
\end{itemize}

This pattern suggests that cross-field tool discovery
is one principal bottleneck for current AI systems in
research-level mathematics.

The second failure mode is illustrated by P01:
\textbf{domain-specific regularity blindness}.  Hairer
confirmed that our proof's first non-trivial statement---that
the renormalized cubic is $O(1)$---is wrong.  The Wick-ordered
cube in $\Phi^4_3$ is a distribution in a negative regularity
space, not a bounded function.  This error is characteristic of
AI systems generating plausible-sounding mathematical statements
that are false in the specific technical regime of the problem.
Critically, a non-expert orchestrator has no way to catch such
errors, and the AI systems themselves did not flag the claim as
problematic during cross-checking.  Gubinelli further observed
that the models misidentified the mechanism of the proof
(fixating on mass renormalization, which was chosen for
convenience) and that the models' own Section~7 correctly notes
that the variance of the renormalized cube diverges---directly
contradicting the Section~6 claim that it is~$O(1)$.  The models
generated both the correct fact and the incorrect claim in
different sections of the same document without recognizing the
contradiction.  This represents a failure mode that is arguably
more dangerous than the cross-field discovery gap, because the
proof reads as superficially coherent.

\subsection{Comparison with single-prompt LLM baselines}

The official commentary~\cite[\S4]{FirstProofSolutions2026}
describes outputs from Gemini~3.0 Deep Think and ChatGPT~5.2
Pro tested on February~4--5, 2026 in single-prompt mode.
The comparison is informative:

\begin{itemize}[nosep]
\item On P01, single-prompt LLMs quoted Hairer's note without
  proof or assumed absolute continuity $\mu \sim \mu_0$
  (false).
\item On P02, LLMs constructed test vectors depending on~$\pi$,
  violating the universality requirement.
\item On P03, LLMs proposed Metropolis--Hastings chains
  (trivial, not satisfying the problem's nontriviality
  condition) or confused the interpolation and
  non-interpolation Macdonald polynomials.
\item On P05, LLMs stated the correct theorem but proofs were
  ``sketched or slightly garbled''~\cite[\S4.5]{FirstProofSolutions2026}.
\item On P07, all single-prompt LLM proofs contained false
  lemmas.
\end{itemize}

Our iterative multi-session approach avoided these failure
modes in every case where the single-prompt baselines failed,
suggesting that sustained multi-turn reasoning with
cross-model verification is a significant factor in proof
quality.  However, on the three problems where we produced
incomplete results (P04, P06, P07), the single-prompt
baselines also failed, indicating that these problems pose
fundamental challenges for current LLM architectures
regardless of prompting strategy.

\subsection{Role of the human operator}

A distinctive feature of this work is that the human operator
(the author) is not a mathematician.  The author's
contribution was purely organizational: selecting which AI
model to engage, routing outputs between models, and
maintaining workflow momentum.  Mathematical strategy,
correctness evaluation, and proof pivots were handled entirely
by the AI systems.

This raises the question of whether the results should be
attributed to the AI systems, the human orchestrator, or the
collaborative process.  We take no position on this question
but note that the same AI systems, when used in single-prompt
mode by the challenge organizers, produced substantially
weaker results (\S\ref{sec:discussion}).  The iterative
multi-model workflow appears to be a necessary component, not
merely a convenience.

\subsection{Limitations of this analysis}

Several limitations should be noted.  First, our assessment of
proof completeness is based on our own reading of the proofs
and the official authors' commentary; we have not submitted
our proofs for independent peer review.  Second, the
comparison with single-prompt baselines is indirect: the
organizers tested different model versions on different dates
under different conditions.  Third, this paper was itself
produced using AI assistance (as disclosed in the abstract),
which introduces the possibility of systematic blind spots in
self-assessment.

%% ============================================================
\section{Conclusion}\label{sec:conclusion}

We have presented a systematic comparison of AI-assisted
solutions to the ten First Proof challenge problems with the
official human-generated solutions.  The principal findings are:

\begin{enumerate}[nosep]
\item Five of ten problems received correct answers with
  complete proofs, often via methodologically distinct
  approaches.
\item The four problematic results reveal two failure modes:
  (a)~cross-field tool discovery (P04, P06, P07), and
  (b)~domain-specific regularity blindness---generating
  plausible but false claims about the technical properties
  of mathematical objects (P01).
\item Iterative multi-model collaboration avoided the failure
  modes observed in single-prompt LLM baselines, but did not
  overcome either bottleneck.
\item A non-expert human operator can orchestrate AI systems
  to produce research-level mathematical proofs, but cannot
  independently verify correctness.  The P01 error was only
  caught by the problem author himself.
\end{enumerate}

An initial automated review categorized 8/10 of our solutions
as correct or superior.  On critical re-examination against the
official solutions and direct author feedback, this assessment
is significantly inflated: the honest count is 5/10 complete,
1/10 valid approach, 1/10 incorrect proof (correct answer),
2/10 incomplete, and 1/10 divergent.  We report this discrepancy
as a cautionary note on the reliability of AI self-assessment in
mathematical contexts.

%% ============================================================
\begin{thebibliography}{99}

\bibitem{FirstProof2026}
M.~Abouzaid, A.~Blumberg, M.~Hairer, J.~Kileel, T.~Kolda,
P.~Nelson, D.~Spielman, N.~Srivastava, R.~Ward, S.~Weinberger,
and L.~Williams.
\newblock First Proof: Ten research-level mathematical problems.
\newblock \url{https://1stproof.org/}, arXiv:2602.05192, 2026.

\bibitem{FirstProofSolutions2026}
M.~Abouzaid et~al.
\newblock First Proof solutions and comments.
\newblock Released February~14, 2026.
\newblock \url{https://codeberg.org/tgkolda/1stproof/}

\bibitem{AlphaProof2024}
DeepMind.
\newblock AI achieves silver-medal standard solving International
  Mathematical Olympiad problems.
\newblock Blog post, July 2024.
\newblock \url{https://deepmind.google/discover/blog/ai-solves-imo-problems-at-silver-medal-level/}

\bibitem{AlphaGeometry2024}
T.~Trinh, Y.~Wu, Q.~Le, H.~He, and T.~Luong.
\newblock Solving Olympiad geometry without human demonstrations.
\newblock \emph{Nature}, 625:476--482, 2024.

\bibitem{FunSearch2024}
B.~Romera-Paredes et~al.
\newblock Mathematical discoveries from program search with large
  language models.
\newblock \emph{Nature}, 625:468--475, 2024.

\bibitem{LLMEnsemble2025}
S.~Jiang, D.~Kadosh, and N.~Shazeer.
\newblock Towards large reasoning models: A survey of reinforced
  learning for mathematical reasoning.
\newblock arXiv:2501.09686, 2025.

\bibitem{MoA2024}
J.~Wang et~al.
\newblock Mixture-of-agents enhances large language model
  capabilities.
\newblock arXiv:2406.04692, 2024.

\bibitem{AMW25}
A.~Ayyer, J.~Martin, and L.~Williams.
\newblock The inhomogeneous $t$-PushTASEP and Macdonald polynomials
  at $q=1$.
\newblock \emph{Ann.\ Inst.\ Henri Poincar\'e D}, 2025.

\bibitem{BGLS01}
H.~Bauschke, O.~G\"uler, A.~Lewis, and H.~Sendov.
\newblock Hyperbolic polynomials and convex analysis.
\newblock \emph{Canad.\ J.\ Math.}, 53(3):470--488, 2001.

\bibitem{BSS12}
J.~Batson, D.~Spielman, and N.~Srivastava.
\newblock Twice-Ramanujan sparsifiers.
\newblock \emph{SIAM J.\ Comput.}, 41(6):1704--1721, 2012.

\bibitem{HKN24}
M.~Hairer, S.~Kusuoka, and H.~Nagoji.
\newblock Singularity of solutions to singular SPDEs.
\newblock arXiv:2409.10037, 2024.

\bibitem{BDW25}
H.~Ben~Dali and L.~Williams.
\newblock A combinatorial formula for interpolation Macdonald polynomials.
\newblock arXiv:2510.02587, 2025.

\end{thebibliography}

\end{document}
